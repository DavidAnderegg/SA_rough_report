During the course of this project, a modification to the Spalart Allmaras (SA)
turbulence model for rough walls has been implemented. The changes have been
proposed by Boeing and published by \cite{sa_rough}. The SA model is a
one-equation model that solves for the eddy viscosity. The distance to the wall
is used for the turbulence length scale. The modification for rough walls simply
changes this distance, meaning it shifts a virtual wall upwards. This increases
the height of the boundary layer which is the primary effect of roughness.

To modify the wall distance, the roughness value of the nearest surface cell is
needed. This might sound easy on first sight. But one has to keep in mind that
ADflow is a parallel code. This means the whole mesh is split across different
processors. Each processors has only access to mesh cells he owns. Thus the
roughness value of the nearest wall for the current cell might live on a
different processor and must be communicated across. Figuring out this
communication was challenging and error prone. To make sure it was implemented
correctly, testcases with cubes and cuboids were set up. Those tests would
either fail or look wrong when the communication was not implemented correctly.

To validate the modifications itself, test cases with rough walls were setup.
Those cases compare ADflow to experimental data, theory and the open source
Solver SU2. They could show that the rough modification with a surface roughness
of 0 would still be the same as the standard SA model. Unfortunately, they also
made it clear that current implementation has a bug where the skin friction
coefficient and the velocity profile are wrongly shifted for rough surfaces.
This means, the predicted shape looks right, but the predicted values are not.
It is unclear what is causing this. It will be investigated in the future.

The before mentioned test cases also showed that the expected grid convergence
rate of 2 was not achieved. The real value lies more around 1. This will also be
investigated further.

Lastly, automated tests were setup to make sure future changes do not break
existing features. This concept is know as regression tests. In ADflow, the
automated tests are also mixed with unit tests which prove a new feature is
working. In this context, the gradients of the modified SA model was compared
against complex step. Unfortunately, they do not completely agree and are off by
a relative tolerance of \num{1e-7}. It is unclear what was causing this and will
be investigated further.

To conclude, a solid base was created. Although some bugs are still present that
need further work. Luckily, they do not affect an optimization in the following
sense: (1) The lower than expected grid convergence does not directly affect an
optimization. (2) The offset of the gradients is so small that an optimization
would still converge, maybe a bit slower. (3) The offset of the roughness
effects can be compensated with a higher than expected surface roughness value.
This is possible because the predicted shape of the roughness effects match.
And lastly, in optimization, it is more important to take effects of roughness
into account than accurately predicting those. After all, it is extremely
difficult to define a valid roughness value for e.g a soiled wind turbine blades
after 20 years of service.
