In a previous project, the author developed new airfoils for a vertical axis
wind turbine \cite{awp} using a CFD Solver called ADflow. It became clear that
the performance of the airfoils in soiled conditions must be taken into
account. The airfoils are highly sensitivity to surface contamination if
neglected. To fix this, the author implemented a `hacky` variant of the rough
\gls{sa} turbulence model.


\section{Goals}
The goal of this project is to streamline and finalize this implementation.
Namely the following sub-goals should be achieved:

\begin{itemize}
  \item Modify the existing \gls{sa} turbulence model for rough walls
  \item Test and verify the implementation.
  \item Use \gls{ad} to differentiate the newly added code
  \item Test and verify the gradients for the rough variant
\end{itemize}


\section{ADflow}
The CFD Solver ADflow (AD stands for \Gls{ad}) is developed at the university of
Michigan. It is based on a structured, multiblock solver called `sumb`. It has
been adapted for gradient based optimization by means of \gls{ad} and the
`adjoint method`. Additionally it has some highly efficient and robust ANK and
NK Solvers. Those are needed to achieve machine-precision convergence which is a
requirement for efficient gradient basded optimization \cite{Mader2020a}
\cite{Kenway2019a} \cite{Yildirim2019b}.

ADflow has a python interface which makes it fairly easy and straight forward to
be used in optimizations scripts. But the heavy lifting is done in Fortran. This approach allows to combine highly effiecent and fast computations with the easy of a scripting language.
